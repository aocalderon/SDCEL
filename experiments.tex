\section{Experiments}

\subsection{Testing overlay alternative methods}
To evaluate the best alternatives during the reduce/merge stage, three different approaches were evaluated to glue together segments of faces which could be located in different partitions.  As it was mentioned earlier, those closed segments which form auto-contained faces inside of each partition are immediately reported and there is no need for additional processing.  However, those segments which touch the border of the cell in their partition must be post-processed to evaluate if they could be extended with segments in the neighborhood partitions.

The first naive approach was to collect those segments touching the borders in a master/root node which sequentially will combine segments with the same label and concatenate them accordingly in order to create the final face.  It is a straightforward method but could be really costly if the number of partitions and the subsequent number of segments touching borders increase.

As an alternative, it is proposed to do an intermediate step with a parameter introduced by the user were a level in the quadtree structure is given.  Following this parameter, the segments in partitions below the given level are collect them in intermediate nodes of the quadtree were can be evaluated partially.  The goal of this step is that part of the work can be distributed in a larger number of nodes as a previous step before to be sent to the master/root node for final evaluation.  It is expected that mos of the work can be done in this intermediate step.  However, those partitions which lie above of the user defined level still have to be evaluated in an individual node.

It is clear that it will create an optimization issue, if we choose a level down in the structure it can be evaluated in a larger number of intermediate nodes but, at the same time, a large number of partitions will lie above of that threshold and they will be evaluated by an unique node which can dominate the execution time.  On the other hand, if a level is selected high in the quadtree, just a few number of partitions will have to be evaluated in that unique node, but the number of intermediate nodes will also be reduce and the execution time for their evaluation will increase.

\begin{figure}[!ht]
    \centering
    \includegraphics[width=\linewidth]{figures/experiments/Overlay_Tester}
    \caption{Overlay methods evaluation.}\label{fig:overlay_tester}
\end{figure}

\subsection{Real datasets}
In order to test our implementation, we used two datasets of polygons.  The first one is the California Census-Track Administrative Levels - CCTAL. The layer A collects the census tracts in California during 2000 meanwhile the layers B collects those ones for 2010.  Something to note with this dataset is that the differences between the layers no just show the changes in time but also they present a small gap (due to technical reasons) between the polygons which will increase considerably the number of intersections and new faces the DCEL should be able to identify.

The second dataset is from the Global Administration Areas - GADM\footnote{\url{https://gadm.org/}}. It collects geographical boundaries of the countries and their administrative divisions around the globe.  For the tests, we select the levels 0 (Countries) and level 1 (States). We take individual polygons, it is, any multi-polygon was divided and each polygon was manage independently. The details of both datasets are summarized in table \ref{tab:datasets}.  Experiments runs on a cluster of 12 nodes running Apache Spark 2.4.  Each node has 9 cores and 12G memory available.

\begin{table}[!ht]
    \caption{Datasets details.}
    \label{tab:datasets}
    \begin{tabular}{c c c c}
        \toprule
        Dataset & Layer & Number        & Number    \\
                &       & of polygons   & of edges  \\
        \midrule
        CCTAL & Polygons for 2000 & 7028 & 999763  \\
              & Polygons for 2010 & 8047 & 2901758 \\
        GADM  & Polygons for Level 0 & 116995 & 32789444 \\
              & Polygons for Level 1 & 117891 & 37690256 \\
        \bottomrule
    \end{tabular}
\end{table}

\subsection{California dataset}

The first set of experiments aim to compare the performance of the SDCEL implementation over the CCTAL Dataset varying the number of partitions used to divide the data. It sets the parameters of the quadtree used to split the data to obtains different values from coarse (100) to fine (15K) number of partitions (left part in figure \ref{fig:ca}).  Clearly there are a trade-off in this case.  The more partitions, each subdivision will have to deal with a smaller number of edges will means a faster execution.  However, it the number of partitions is too high, the cost to divide and collect the results will have an impact.  We can see that an optimal value can be reached around 7K partitions.  A zoom of this part of the plot is shown in figure \ref{fig:ca}.

In addition with the partition analysis, we compare the performance of the SDCEL implementation against the sequential solution offered by the CGAL library.  The independent column in the right of figure \ref{fig:ca} shows its execution time.  Clearly, the appropriate setting for the number of partitions of SDCEL can beat the sequential implementation by several orders of magnitude.

\begin{figure}[!ht]
    \centering
    \subfloat[SDCEL vs CGAL. \label{fig:ca_a}]{%
        \includegraphics[width=0.45\linewidth]{figures/experiments/ca.pdf}}
    \hfill
    \subfloat[Focus on the most relevant number of partitions. \label{fig:ca_b}]{%
        \includegraphics[width=0.45\linewidth]{figures/experiments/ca_sample.pdf}}
    \caption{Experiments with CCTAL dataset. \label{fig:ca}} 
\end{figure}

\subsection{GADM dataset}

\begin{figure}[!ht]
    \centering
    \subfloat[SDCEL execution time. \label{fig:gadm_a}]{%
        \includegraphics[width=0.45\linewidth]{figures/experiments/gadm.pdf}}
    \hfill
    \subfloat[Focus on the most relevant number of partitions. \label{fig:gadm_b}]{%
        \includegraphics[width=0.45\linewidth]{figures/experiments/gadm_sample.pdf}}
    \caption{Experiments with GADM dataset. \label{fig:gadm}} 
\end{figure}

The CCTAL dataset is relatively small but useful in order to compare with the sequential alternative.  However, the CGAL library is unable to deal with big spatial data.  For the case of the GADM dataset, we only present the results for SDCEL given that CGAL crash when run this volume of data.  Figure \ref{fig:gadm} show the execution time of SDCEL varying the number of partitions.  We can seen a similar trade-off as with CCTAL dataset which can be seen more clearly in the right of the figure.  We can identify a optimal number of partitions around the 6K -8K partitions.  Similar as before, the independent column at the right of figure \ref{fig:gadm} shows the comparison if the SDCEL runs using just one core.

\subsection{Speed up and scale up analysis}

Finally we analyze the scalability of the implementation through the speed up and scale up tests.  For the speed up, the GADM dataset was used with different amount of resources, in this case the number of available nodes.  Each time, the nodes were duplicated and the performance was measured.  Figure \ref{fig:speedup} shows  how as resources double, the response time is almost cut in half each time, as it is expected.  

In the case of the scale up test, the workload was also modified.  The GADM dataset was split in 4 regions slightly similar in the number of edges they collect.  At each iteration, both the size of the data and the amount of available resources were doubled. As it was expected, the performance at each scenario should not change significantly as it is shown in figure \ref{fig:scaleup}.

\begin{figure}[!ht]
    \centering
    \subfloat[Speed Up \label{fig:speedup}]{%
        \includegraphics[width=0.45\linewidth]{figures/experiments/speedup.pdf}}
    \hfill
    \subfloat[Scale Up \label{fig:scaleup}]{%
        \includegraphics[width=0.45\linewidth]{figures/experiments/scaleup.pdf}}
    \caption{Speed Up and Scale Up experiments.} \label{fig:speed_scale} 
\end{figure}
