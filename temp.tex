\section{Removed from Literature...}
The data explosion we are living today has impacted not just common areas of our daily lives such as our health, economy or social relationships but also how we move and how we locate ourselves in space.  Nowadays, huge amount of information with a spatial component are generated.  For example, GPS devices and smart-phones are able to collect your location and support navigation queries.  Similarly, earth observation satellites captures gigabytes of imagery all around the world at daily basis.

This new availability has brought novel and interesting applications but also it is demanding most difficult challenges.  The special characteristics of spatial data (i.e. topology, precision issues, types of data) demands special techniques, particularly when we are dealing with large and heterogeneous datasets.  The study of efficient algorithms to deal with spatial data has touched diverse domains such as computer sciences, mathematics and physical geography.  In one of these aspects, the computational geometry has explored different techniques and data structures to support such efficiency.

In this area, one of the challenges is related to spatial operations between layers of geographic data.  One of the most commons outputs of geographic information systems are maps.  They can provide interesting information in an agile and visual manner, and it is not a surprise than many areas are interested on performing further analysis using maps as an input.  For example, ecology map users are usually interested on discovering areas where different attributes co-exist, for instance, the presence of certain vulnerable animal species and the impact of road construction or accessibility to water resources.  

This kind of queries are supported by binary operators such as intersection, union and difference.  They are implemented in different ways and their applications are widely reported in literature.  However, most of this solutions are based on and supported on top of edge-list data structures.  Perhaps, the most common implementation of this kind of structures is the DCEL (Double Connected Edge List).  The solution of overlay operations and the use of DCEL data structures are tightly coupled.

