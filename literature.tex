\section{Literature review}
-- Not here
The data explosion we are living today has impacted not just common areas of our daily lives such as our health, economy or social relationships but also how we move and how we locate ourselves in space.  Nowadays, huge amount of information with a spatial component are generated.  For example, GPS devices and smart-phones are able to collect your location and support navigation queries.  Similarly, earth observation satellites captures gigabytes of imagery all around the world at daily basis.

This new availability has brought novel and interesting applications but also it is demanding most difficult challenges.  The special characteristics of spatial data (i.e. topology, precision issues, types of data) demands special techniques, particularly when we are dealing with large and heterogeneous datasets.  The study of efficient algorithms to deal with spatial data has touched diverse domains such as computer sciences, mathematics and physical geography.  In one of these aspects, the computational geometry has explored different techniques and data structures to support such efficiency.

In this area, one of the challenges is related to spatial operations between layers of geographic data.  One of the most commons outputs of geographic information systems are maps.  They can provide interesting information in an agile and visual manner, and it is not a surprise than many areas are interested on performing further analysis using maps as an input.  For example, ecology map users are usually interested on discovering areas where different attributes co-exist, for instance, the presence of certain vulnerable animal species and the impact of road construction or accessibility to water resources.  

This kind of queries are supported by binary operators such as intersection, union and difference.  They are implemented in different ways and their applications are widely reported in literature.  However, most of this solutions are based on and supported on top of edge-list data structures.  Perhaps, the most common implementation of this kind of structures is the DCEL (Double Connected Edge List).  The solution of overlay operations and the use of DCEL data structures are tightly coupled.
-- Not here

The fundamentals of the DCEL are explained in the seminal paper of \cite{muller_finding_1978} and illustrated with more examples in \cite{preparata_computational_1985}.  The authors highlight among the main advantages of DCELs the opportunity of capturing topological information and allowing multiple overlay operations once the DCEL is created.  In addition, a DCEL can be constructed in $\mathcal{O}(n log(n))$ time using $\mathcal{O}(n)$ additional memory where $n$ is the number of vertices in the input layer \cite{freiseisen_colored_1998}. Once created, the DCEL allows Boolean overlay operations in $\mathcal{O}(n)$ time using $\mathcal{O}(n)$ additional space. 

Nowadays, implementations of DCEL data structures have been presented and used in diverse applications.  \cite{barequet_dcel_1998} presented the description and modelling of a database-oriented implementation of a DCEL. It presents a geometric package aims to support polyhedra or polyhedral surface operations.  The design is discussed in detail together with application examples for collision detection, gap filling and inter-slice interpolation.

In \cite{freiseisen_colored_1998}, the author explains in more detail the different spatial operations (intersection, union and difference) supported by a DCEL while describes a sequential implementation over the Computational Geometry Algorithms Library - CGAL.  This work allows, given two simple polygon layers, the intersections of their boundaries keeping track of related attributes of the inputs.  

A cardinal reference for DCEL description and design should be \cite{berg_computational_2008}.  It describes the core of the DCEL construction algorithm but also go deeper in the explanation of related concepts and techniques used during the DCEL generation such as line segment intersection and plane sweeping methods.  It also compiles a number of well-explained examples and applications of its usage, for instance, triangulations, point location and robot motion planning.

\cite{boltcheva_topological-based_2020} describes the use of sequential DCELs to store the needed topological information of mesh networks extracted from LiDAR data to reconstruct roof polygons in 2D.  They highlight the usefulness of 2D topological data structures to avoid the complexities in data storage and handling of 3D configurations.

Although there is not reference to distributed DCEL implementations, other dynamic parallel data structures has been described.  For example, the DD-Rtree \cite{challa_dd-rtree_2016} claims to preserve spatial locality while distributing data across compute nodes.  By contrast to static counterparts, it can be constructed incrementally making it useful for handling big data.  It evaluates its communication cost, query time and data performance of data mining algorithms.  In addition, this work provides an interesting survey of multi-dimensional indexing structures and their parallel versions.  Similarly, \cite{sabek_spatial_2017} describe different techniques to support distributed spatial joins, including options to perform spatial partitions in layers previous to the join phase.  Together with this, it evaluates cost-based and rule-based query optimization.  

However, spatial indexes and spatial joins could support overlay operations in certain way but just for an individual operator at a time. 

\cite{li_scalable_2019} also explore parallel support for common computational geometry operations including polygon union, convex hull and skylines.  They apply the MapReduce paradigm over the Hadoop framework presenting the CG\_Hadoop tool.  However, as before, it just supports limited number of operations over an individual input layer once at a time.

\cite{franklin_data_2018} also present a survey of parallel spatial algorithms but it focus on GPU and multi-core architecture.  They mention works related to hierarchical tree structures (octree or rtree), location point algorithms, line and plane sweep algorithm, cluster-based parallel map overlays, among others.  Also, they mention the use of those techniques in the solution of diverse operations such as: union of large set of polygons, planar graphs, overlaying maps, 3D overlay triangulation and cross areas in overlay polygons.

\cite{magalhaes_fast_2015} performs the map overlay in parallel, thereby utilizing the ubiquitous multi-core architecture.  It uses a two-level grid partitioning strategy and proposed a conservative empirical formula for the grid size that gave a good execution time and a feasible memory size.  Together with this, they implement exact computation using rational number to overcome round-off errors. As result, they provide the EPUG-Overlay package to support intersection operations.

\cite{puri_efficient_2013} and \cite{puri_mapreduce_2013} revisit the distributed polygon overlay problem and its implementation using the GPGPU model and the MapReduce paradigm respectively.  \cite{puri_mapreduce_2013} present the adaptation and implementation of a polygon overlay algorithm and describe the system to execute a distributed version on a Linux cluster using Hadoop MapReduce framework.  However, they only analyze and report intersection overlay operation since it is the most widely used and representative operation.  They ported an MPI based spatial overlay system (GPC library) to Hadoop MapReduce platform and a grid based overlay algorithm with two alternatives: a single map and reduce phase, and a map phase only using distributed caches.

In \cite{puri_efficient_2013}, the research focuses to develop an array based parallel overlay algorithm which can be easily implemented on GPUs using prefix sum and sorting to,  potentially, speedup overlay computation.  In this approach, they use a distributed Rtree construction using a top-down schema and a load-balanced overlay processing system using the MapReduce paradigm.  The proposed parallel algorithm uses a similar idea of a DCEL with the representation of a new set of polygons by the merge of edges from the initial layers and their intersections.  However it uses scan lines to partition the set of edges and process them concurrently. As the previous work, they test the implementation only with the intersection overlay operation.

Theses previous works focus on solutions aim to multi-core architectures or GPGPU models which scalability can be affected by very large datasets.  Even though, implementations over the Hadoop ecosystem could not take total advantage of modern distributed memory frameworks such as Apache Spark.

Currently, there are just a few sequential implementations available in the market.  The most important are LEDA\footnote{\url{https://www.algorithmic-solutions.com/}} \cite{mehlhorn_leda_1995}, Holmes3D\footnote{\url{http://www.holmes3d.net/graphics/}} \cite{holmes_dcel_2021} and CGAL\footnote{\url{https://www.cgal.org/}} \cite{fogel_cgal_2012}.  LEDA and Holmes3D are close-source software and their access is limited.  On the other hand, CGAL is an open-source project with a large trajectory in the area of computational geometry offering a wide-ranging number of packages and modules to support diverse areas from modular arithmetic to geometric optimization.

CGAL offers a solid support for DCEL construction and overlay operations on top of it through the Arrangement package.  Its design concepts, implementation and well documented applications are available by different sources (\cite{flato_design_2001}, \cite{haran_experimental_2009} and \cite{wein_advanced_2007}).  \cite{fogel_cgal_2012} deserves a particular entry.  It does just not demonstrate the features of the DCEL implementation but also it collects a large number of application in diverse and novel areas like computer-assisted surgery and molecular biology.
