\section{Introduction}
The use of spatial data structures is ubiquitous in many areas ranging from computational geometry, robotics and geographic information systems.  It has important advantages to offer to current spatial algorithms in particular to apply spatial indexes and opportunities to exploit proximity among the data.  Many works depends on it to ensure efficiency for example spatial joins, voronio diagrams and robot motion planning (Berg citation).

Although spatial data structure such as grids, the different flavours of rtrees and quadtrees are widely reported in literature, other structures have been less the focus of attention.  In particular, edge-list structures appear as a technique used in diverse application, specially related with geometric computation.  The use of such structure has been reported in application such as: obtaining silhouettes of polyhedra, efficient Minkowksi sums and offset of polygons and thematic map overlay. 

The most representative data structure in the edge-list family is the Double Connected Edge List (DCEL).  A DCEL is a data structure which collect topological information of the edges, vertices and faces contained by a surface in the plane.  The DCEL and its components represent a planar subdivision of that surface in the plane. In a DCEL, the faces (polygons) represents the cells of the subdivision; the edges are boundaries which divide adjacent faces; and the vertices, itself, are boundaries between adjacent edges.

\vspace{1cm}
Still working on...
\begin{itemize}
    \item Problem in the current DCEL...
    \item What is our proposal...
    \item Contributions...
\end{itemize}
