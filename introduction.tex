\section{Introduction}
The use of spatial data structures is ubiquitous in many areas ranging from computational geometry, robotics and geographic information systems.  It has important advantages to offer to current spatial algorithms in particular the application of spatial indexes and opportunities to exploit proximity among the data.  Many works depends on it to ensure efficiency for example spatial joins, voronio diagrams and robot motion planning \cite{berg_computational_2008}.

Although spatial data structures such as grids and the different flavours of rtrees and quadtrees are widely reported in literature, other structures have been less the focus of attention.  In particular, edge-list structures appear as a technique used in diverse applications, specially related with geometric computation, but their employment have been limited to very specific domains.  The use of such kind of structures has been reported in applications like obtaining silhouettes of polyhedra, efficient Minkowksi sums and offset of polygons and diverse types of triangulations mainly in computational geometry projects.

The most representative data structure in the edge-list family is the Double Connected Edge List (DCEL).  A DCEL \cite{muller_finding_1978, preparata_computational_1985} is a data structure which collect topological information of the edges, vertices and faces contained by a surface in the plane.  The DCEL and its components represent a planar subdivision of that surface in the plane. In a DCEL, the faces (polygons) represents the cells of the subdivision; the edges are boundaries which divide adjacent faces; and the vertices, itself, are boundaries between adjacent edges.

DCEL data structures have proved its utility in different aspects.  For instance, in computational geometry, the use of connected edge lists is cardinal to support polygon triangulations and their applications in surveillance (the Art Gallery Problem \cite{chvatal_combinatorial_1975, orourke_art_1987}) and robot motion planning (Minkowski sums \cite{berg_computational_2008, chew_convex_1993}).  DCELs are also used to perform polygon union on printed circuit boards to support the simplification of connected components in an efficient manner \cite{fogel_cgal_2012} and the computation of silhouettes from polyhedra \cite{fogel_cgal_2012, berberich_arrangements_2010} applied frequently in computer vision and 3D graphics modelling \cite{boguslawski_modelling_2011}.

One interesting problem in which edge-list data structures have shown utility are the thematic overlay maps. In this problem, two polygon input layers capture geospatial information and attribute data for different kind of phenomena.  In many areas, such as ecology, economics and climate change, it is important to be able of join those layers and match their attributes in order to unveil patterns or anomalies in data which can be highly impacted by location.  Several operations are relevant depending of the study, for instance, sometime the user would like to find intersections between the layers and other times be able to see the difference between them. 

Even the DCEL data structure has interesting advantages for overlay map operations, current methods are sequential based.  For layers with thousand of polygons the execution time is not feasible, taking days to get some results.  Further more, most of distributed techniques that have been used in this matter are oriented to a specific spatial operation (intersection, union or difference) and they have to be run from scratch if other operation is required.  In addition, current parallel techniques divide the data into partitions and replicate features if needed in order to solve the problem locally.  It could potentially increase the size of the problem.  Up to now, data structures collecting topological properties, like the DCEL, are not explored in a distributed and scalable fashion in order to support typical overlay operators.

Even sequential DCEL implementation are not new, nowadays, with the scale and volume of available geodata, the rise of big (spatial) data makes necessary to count with fast and efficient techniques for spatial analysis.  For example, today GIS researchers have to deal with spatial operations between layers collecting thousand of counties boundaries at nation-wide level.  The versatility and efficiency of the spatial methods is cardinal for their studies. Given the advantages shown by the DCEL data structure, it should be interesting to count with intermediate data structures that not just exploit the advantages of distributed frameworks but also allow multiple map overlay queries at the same time. 

In addition, we already know that current topological data structures are common in computational geometry.  However, most implementations are sequential and they do not scale appropriately on large spatial datasets.  Moreover, distribute alternatives like parallel spatial joins add complexity due to spatial partitioning and replication and they are unable to run more than one operation once they have been created.

In this paper we describe the design and implementation of a scalable and distributed DCEL data structure.  To do that, we present a partition strategy that ensure a correct split of the input data.  It guarantees that each partition collect the required data to work independently minimizing duplication and transmission costs.  In addition, the partition strategy presents a merging procedure that guarantee the collection of all the individual results and the consolidation of the final and correct answer.  In section \ref{sec:strategy} we present and discuss the implementation of a novel strategy to divide the input data and build a distribute and scalable DCEL to be used in a parallel framework such as Apache Spark. 

The implementation of the distributed DCEL arises some interesting pitfalls.  First, the proposal must be ready to deal with potential anomalies which are not present in the sequential execution.  For example, the implementation should apply an appropriate management of features such as holes and multi-polygons which could lay on different partitions.  Such features could potentially lost the reference of their other components compromising the correctness of the implementation.  Appropriate mechanism should be applied to solve these possible cases.

Secondly, once a scalable and distributed DCEL has been built, it must support a set of Boolean overlay operations in a transparent manner.  It is, the query of such operators should take advantage of the scalability of the DCEL and be able to run also in a parallel fashion.  Additionally, the different operators should be able to be run multiple times without the need of rebuild the data structure.  That means, that the Boolean operators should be applied on top of the distributed DCEL in a parallel fashion and , once it is done, the local results should be integrated together in order to remove any duplication and merging partial results obtained on contiguous partitions.

The contributions of this work aim to the solution of the above-mentioned challenges.  In section \ref{sec:anomalies} we address the complexities that the novel partitioning schema could arise.  We integrate a solution to solve cases such as isolated holes or empty partitions which could lost reference data during the distribution. We present an algorithm to detect and track partitions with orphans components and systematically search their neighborhood for related features.  

Section \ref{sec:overlay} provides the details of the implementation of the Boolean overlay operators which query the DCEL.  The operations for intersection ($A \cap B$), union ($A \cup B$),  difference ($A - B$) and symmetric difference ($A \triangle B$) are available.  They are executed locally at each partition of the DCEL and then collecting and merging the results to provide the final answer.  The parallel execution take advantage of the distribution and efficiently exploits the locality to present the results.

At the best of our knowledge, there is not a distributed and scalable DCEL implementation available that meet those criteria.  We think it could be a great tool to support key challenges and operations in Geoscience today.